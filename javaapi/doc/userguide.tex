%
%LaTeX 2.e output generated byt AFT
%
\documentclass[11pt]{article}
\usepackage{ae}
\usepackage[T1]{fontenc}
\usepackage[latin1]{inputenc} % latin1 encoding
\usepackage{graphicx}         % graphic files
\usepackage{alltt}            % filtered verbatim
\usepackage[margin=2.5cm]{geometry} % page layout
\usepackage[colorlinks,urlcolor=blue]{hyperref} % URLs an hrefs
%
\setlength{\parindent}{0pt}
\setlength{\parskip}{3pt}
\pagestyle{headings}
%
\begin{document}

\title{Using the MetaMap Java API}
\author{Willie Rogers}
\maketitle

\section{Purpose}
\label{Purpose}

MetaMap maps terms occuring in text to UMLS Metathesaurus concepts. As
part of this mapping process, MetaMap tokenizes text into sections,
sentences, phrases, terms, and words. MetaMap maps the noun phrases of
the text to the best matching UMLS concept or set of concepts that
best cover each phrase.  The MetaMap Java API provides java programs
with programmatic access to MetaMap mapping engine.


\section{MetaMap API's Underlying Architecture}
\label{MetaMap API's Underlying Architecture}

MetaMap mapping engine is written primarily in Quintus Prolog; to
facilitate its use by Java programs, the system uses PrologBeans to
provide a loose coupling between the Java API and the mapping engine.
See the Quintus Prolog PrologBeans documentation for more information
( \url{http://www.sics.se/isl/quintus/html/quintus/pbn.html}).


\section{Pre-requisites}
\label{Pre-requisites}

The full MetaMap download and installation is required to use the
MetaMap Java API (see \(\backslash\)http://metamap.nlm.nih.gov/\#Downloads).  Also,
Java 1.6 SDK or greater is required.  (It should work Java 1.5 but it
has not been tested with Java 1.5)


\section{Downloading, Extracting and Installing the API distribution}
\label{Downloading, Extracting and Installing the API distribution}

In the directory where you installed the Public Metamap (the directory
containing the public\_mm directory) extract the javaapi archive:

\begin{verbatim}
$ bzip2 -dc /home/piro/public_mm_linux_javaapi_2010.tar.bz2 | tar xvf -
\end{verbatim}

If you plan on modifying the sources to the prolog-based MetaMap
server (mmserver) you will need to download and extract the source
archive
( \url{http://metamap.nlm.nih.gov/download/public\_mm\_src\_2010.tar.bz2}) as
well:

\begin{verbatim}
$ bzip2 -dc /home/piro/public_mm_src_2010.tar.bz2 | tar xvf -
\end{verbatim}

You will need to re-run ./bin/install.sh from the public\_mm directory
to setup the files for javaapi.

\begin{verbatim}
$ ./bin/install.sh
\end{verbatim}

\section{Using the metamap server}
\label{Using the metamap server}

\subsection{Starting supporting servers}
\label{Starting supporting servers}

The metamap server (mmserver) must first be running to use the Java
API.  If the SKR/Medpost Tagger is not already running start it using
the following command:

\begin{verbatim}
$ ./bin/skrmedpostctl start
\end{verbatim}

If you wish to the Word Sense Disambiguation (WSD) Server (optional),
start it also.

\begin{verbatim}
$ ./bin/wsdserverctl start
\end{verbatim}

\subsection{Running the metamap server}
\label{Running the metamap server}

Then start the metamap server:

\begin{verbatim}
$ ./bin/mmserver10
/home/piro/public_mm/bin/SKRrun -L 2010 \
     -w /home/piro/public_mm/lexicon /home/piro/public_mm/bin/mmserver10.BINARY.Linux -Z 10
Server options: [port(8888),accepted_hosts(['127.0.0.1','130.14.111.76','130.14.110.82'])]
Berkeley DB databases (normal 10 strict model) are open.
Static variants will come from table varsan in /home/piro/public_mm/DB/BDB4/DB.normal.10.strict.
Derivational Variants: Adj/noun ONLY.
Accessing lexicon /home/piro/public_mm/lexicon/data/BDB4/lexiconStatic2010.
Variant generation mode: stat>ic.
\end{verbatim}



\section{Testing the API}
\label{Testing the API}

Using another terminal, you can verify that api is running using the
program \texttt{testapi.sh} which takes a query as an argument:

\begin{verbatim}
$ ./testapi.sh laboratory culture
options: []
terms: laboratory culture 
input text: 
 laboratory culture 
Utterance:
 Id: 00000000.tx.1
 Utterance text: laboratory culture 
 Position: (0, 19)
Phrase:
 text: laboratory culture
 Minimal Commitment Parse: [head([lexmatch([laboratory culture]),inputmatch([laboratory,culture]),tag(noun),tokens([laboratory,culture])])]
Candidates:
 Candidate:
  Score: -1000
  Concept Id: C0430400
  Concept Name: Laboratory culture
  Preferred Name: Laboratory culture
  Matched Words: [laboratory, culture]
  Semantic Types: [lbpr]
  MatchMap: [[[1, 2], [1, 2], 0]]
  MatchMap alt. repr.: [[[phrase start: 1, phrase end: 2], [concept start: 1, concept end: 2], lexical variation: 0]]
  is Head?: true
  is Overmatch?: false
  Sources: [MTH, LNC, MDR, NCI, RCD, MEDCIN, CCPSS, SNOMEDCT]
  Positional Info: [(0, 18)]
 Candidate:
  Score: -861
  Concept Id: C0010453
  Concept Name: Culture
  Preferred Name: Anthropological Culture
  Matched Words: [culture]
  Semantic Types: [idcn]
  MatchMap: [[[2, 2], [1, 1], 0]]
  MatchMap alt. repr.: [[[phrase start: 2, phrase end: 2], [concept start: 2, concept end: 2], lexical variation: 0]]
  is Head?: true
  is Overmatch?: false
  Sources: [MTH, PSY, ICNP, LCH, MSH, NCI, CSP]
  Positional Info: [(11, 7)]
 Candidate:
  Score: -861
  Concept Id: C0022877
  Concept Name: Laboratory
  Preferred Name: Laboratory
  Matched Words: [laboratory]
  Semantic Types: [mnob, orgt]
  MatchMap: [[[1, 1], [1, 1], 0]]
  MatchMap alt. repr.: [[[phrase start: 1, phrase end: 1], [concept start: 1, concept end: 1], lexical variation: 0]]
  is Head?: true
  is Overmatch?: false
  Sources: [LNC, MSH, MTH, NCI, RCD, SNOMEDCT, OMIM, LCH, ALT]
  Positional Info: [(0, 10)]
 Candidate:
  Score: -861
  Concept Id: C0220814
  Concept Name: culture
  Preferred Name: Cultural aspects
  Matched Words: [culture]
  Semantic Types: [ftcn]
  MatchMap: [[[2, 2], [1, 1], 0]]
  MatchMap alt. repr.: [[[phrase start: 2, phrase end: 2], [concept start: 2, concept end: 2], lexical variation: 0]]
  is Head?: true
  is Overmatch?: false
  Sources: [MTH, MSH]
  Positional Info: [(11, 7)]
 Candidate:
  Score: -861
  Concept Id: C0430400
  Concept Name: Culture
  Preferred Name: Laboratory culture
  Matched Words: [culture]
  Semantic Types: [lbpr]
  MatchMap: [[[2, 2], [1, 1], 0]]
  MatchMap alt. repr.: [[[phrase start: 2, phrase end: 2], [concept start: 2, concept end: 2], lexical variation: 0]]
  is Head?: true
  is Overmatch?: false
  Sources: [MTH, LNC, MDR, NCI, RCD, MEDCIN, CCPSS, SNOMEDCT]
  Positional Info: [(11, 7)]
 Candidate:
  Score: -861
  Concept Id: C1706355
  Concept Name: CULTURE
  Preferred Name: Culture Dose Form
  Matched Words: [culture]
  Semantic Types: [bodm]
  MatchMap: [[[2, 2], [1, 1], 0]]
  MatchMap alt. repr.: [[[phrase start: 2, phrase end: 2], [concept start: 2, concept end: 2], lexical variation: 0]]
  is Head?: true
  is Overmatch?: false
  Sources: [MTH, NCI]
  Positional Info: [(11, 7)]
 Candidate:
  Score: -861
  Concept Id: C2242979
  Concept Name: Culture
  Preferred Name: Culture
  Matched Words: [culture]
  Semantic Types: [lbpr]
  MatchMap: [[[2, 2], [1, 1], 0]]
  MatchMap alt. repr.: [[[phrase start: 2, phrase end: 2], [concept start: 2, concept end: 2], lexical variation: 0]]
  is Head?: true
  is Overmatch?: false
  Sources: [MTH, SNOMEDCT, SNM, SNMI]
  Positional Info: [(11, 7)]
 Candidate:
  Score: -827
  Concept Id: C1619828
  Concept Name: Laboratories
  Preferred Name: Laboratories
  Matched Words: [laboratories]
  Semantic Types: [inpr]
  MatchMap: [[[1, 1], [1, 1]]]
  MatchMap alt. repr.: [[[phrase start: 1, phrase end: 1], [concept start: 1, concept end: 1], lexical variation: 0]]
  is Head?: true
  is Overmatch?: false
  Sources: [MTH, HL7V3.0]
  Positional Info: [(0, 10)]
Mappings:
 Map Score: -1000
   Score: -1000
   Concept Id: C0430400
   Concept Name: Laboratory culture
   Preferred Name: Laboratory culture
   Matched Words: [laboratory, culture]
   Semantic Types: [lbpr]
   MatchMap: [[[1, 2], [1, 2], 0]]
   MatchMap alt. repr.: [[[phrase start: 1, phrase end: 2], [concept start: 1, concept end: 2], lexical variation: 0]]
   is Head?: true
   is Overmatch?: false
   Sources: [MTH, LNC, MDR, NCI, RCD, MEDCIN, CCPSS, SNOMEDCT]
   Positional Info: [(0, 18)]
$
\end{verbatim}

The source to java class used \texttt{testapi.sh} is in
\texttt{./src/javaapi/sources/gov/nih/nlm/nls/metamap/MetaMapApiTest.java}.


\section{Using the API}
\label{Using the API}

\subsection{Instantiating the API}
\label{Instantiating the API}

The following sections expose the source code used to produce the
example output shown in previous section

\begin{verbatim}
MetaMapApi api = new MetaMapApiImpl();
\end{verbatim}

If one is running the MetaMap server (mmserver) of a machine other
than the one the Java api client is running, one can specify the
hostname of the MetaMap server when instantiating the api:


If the MetaMap server is running on the host: ``resource.example.org''
then the instantiation would be as follows:

\begin{verbatim}
MetaMapApi api = new MetaMapApiImpl("resource.example.org");
\end{verbatim}

\subsection{Setting MetaMap options}
\label{Setting MetaMap options}
\begin{verbatim}
    List<String> theOptions = new ArrayList<String>();
    theOptions.append("-y");  // turn on Word Sense Disambiguation
    if (theOptions.size() > 0) {
      api.setOptions(theOptions);
    }
\end{verbatim}

Metamap options available in the api: 

\begin{verbatim}
   -@ --WSD <hostname>                   : Which WSD server to use.
   -8 --dynamic_variant_generation       : dynamic variant generation
   -A --strict_model                     : use strict model 
   -C --relaxed_model                    : use relaxed model 
   -D --all_derivational_variants        : all derivational variants
   -J --restrict_to_sts <semtypelist>    : restrict to semantic types
   -K --ignore_stop_phrases              : ignore stop phrases.
   -R --restrict_to_sources <sourcelist> : restrict to sources
   -S --tagger <sourcelist>              : Which tagger to use.
   -V --mm_data_version <name>           : version of MetaMap data to use.
   -X --truncate_candidates_mappings     : truncate candidates mapping
   -Y --prefer_multiple_concepts         : prefer multiple concepts
   -Z --mm_data_year <name>              : year of MetaMap data to use.
   -a --all_acros_abbrs                  : allow Acronym/Abbreviation variants
   -b --compute_all_mappings             : compute/display all mappings
   -d --no_derivational_variants         : no derivational variants
   -e --exclude_sources <sourcelist>     : exclude semantic types
   -g --allow_concept_gaps               : allow concept gaps
   -i --ignore_word_order                : ignore word order
   -k --exclude_sts <semtypelist>        : exclude semantic types
   -l --allow_large_n                    : allow Large N
   -o --allow_overmatches                : allow overmatches 
   -r --threshold <integer>              : Threshold for displaying candidates. 
   -y --word_sense_disambiguation        : use WSD 
   -z --term_processing                  : use term processing 
\end{verbatim}

\subsection{Performing a query using the api}
\label{Performing a query using the api}
\begin{verbatim}
List<Result> resultList = api.processCitationsFromString(terms);
\end{verbatim}

\subsection{Interrogating the result}
\label{Interrogating the result}

\subsubsection{Getting Acronyms and Abbreviations}
\label{Getting Acronyms and Abbreviations}

To get a list of all the acronyms and abbreviations occurring in the
input text use the instance method \texttt{Result.getAcronymsAbbrevs}:

\begin{verbatim}
Result result = resultList.get(0);
List<AcronymsAbbrevs> aaList = result.getAcronymsAbbrevs();
if (aaList.size() > 0) {
  System.out.println("Acronyms and Abbreviations:");
  for (AcronymsAbbrevs e: aaList) {
    System.out.println("Acronym: " + e.getAcronym());
    System.out.println("Expansion: " + e.getExpansion());
    System.out.println("Count list: " + e.getCountList());
    System.out.println("CUI list: " + e.getCUIList());
  }
} else {
  System.out.println(" None.");
}
\end{verbatim}

\subsubsection{Getting Negations}
\label{Getting Negations}

To get a list of all the negated concepts in the input text use the
instance method \texttt{Result.getNegations}:

\begin{verbatim}
List<Negation> negList = result.getNegations();
if (negList.size() > 0) {
  System.out.println("Negations:");
  for (Negation e: negList) {
    System.out.println("type: " + e.getType());
    System.out.print("Trigger: " + e.getTrigger() + ": [");
    for (Position pos: e.getTriggerPositionList()) {
      System.out.print(pos  + ",");
    }
    System.out.println("]");
    System.out.print("ConceptPairs: [");
    for (ConceptPair pair: e.getConceptPairList()) {
      System.out.print(pair + ",");
    }
    System.out.println("]");
    System.out.print("ConceptPositionList: [");
    for (Position pos: e.getConceptPositionList()) {
      System.out.print(pos + ",");
    }
    System.out.println("]");
  }
} else {
	System.out.println(" None.");
}
\end{verbatim}

\subsubsection{Getting Utterances and Associated Phrases, Candidates, and Mappings}
\label{Getting Utterances and Associated Phrases, Candidates, and Mappings}

The instance method \texttt{Result.getUtteranceList()} produces a list of
the utterances present in the result:

\begin{verbatim}
for (Utterance utterance: result.getUtteranceList()) {
	System.out.println("Utterance:");
	System.out.println(" Id: " + utterance.getId());
	System.out.println(" Utterance text: " + utterance.getString());
	System.out.println(" Position: " + utterance.getPosition());
\end{verbatim}

To get the list of phrases, candidates, and mappings associated with
an utterance use the instance method \texttt{Utterance.getPCMList}:

\begin{verbatim}
	for (PCM pcm: utterance.getPCMList()) {
\end{verbatim}

Each phrase, and the list of candidates and mappings associated with
the phrase are encapsulated within a PCM instance.  Use \texttt{PCM.getPhrase}
to get the phrase instance residing within the PCM instance:

\begin{verbatim}
	  System.out.println("Phrase:");
	  System.out.println(" text: " + pcm.getPhrase().getPhraseText());
\end{verbatim}

Similarly, get the candidate list using \texttt{PCM.getCandidateList()}:

\begin{verbatim}
          System.out.println("Candidates:");
          for (Ev ev: pcm.getCandidateList()) {
            System.out.println(" Candidate:");
            System.out.println("  Score: " + ev.getScore());
            System.out.println("  Concept Id: " + ev.getConceptId());
            System.out.println("  Concept Name: " + ev.getConceptName());
            System.out.println("  Preferred Name: " + ev.getPreferredName());
            System.out.println("  Matched Words: " + ev.getMatchedWords());
            System.out.println("  Semantic Types: " + ev.getSemanticTypes());
            System.out.println("  MatchMap: " + ev.getMatchMap());
            System.out.println("  MatchMap alt. repr.: " + ev.getMatchMapList());
            System.out.println("  is Head?: " + ev.isHead());
            System.out.println("  is Overmatch?: " + ev.isOvermatch());
            System.out.println("  Sources: " + ev.getSources());
            System.out.println("  Positional Info: " + ev.getPositionalInfo());
          }
\end{verbatim}

One can get the mappings list from the PCM instance using \texttt{PCM.getMappingList}:

\begin{verbatim}
          System.out.println("Mappings:");
          for (Mapping map: pcm.getMappingList()) {
            System.out.println(" Map Score: " + map.getScore());
            for (Ev mapEv: map.getEvList()) {
              System.out.println("   Score: " + mapEv.getScore());
              System.out.println("   Concept Id: " + mapEv.getConceptId());
              System.out.println("   Concept Name: " + mapEv.getConceptName());
              System.out.println("   Preferred Name: " + mapEv.getPreferredName());
              System.out.println("   Matched Words: " + mapEv.getMatchedWords());
              System.out.println("   Semantic Types: " + mapEv.getSemanticTypes());
              System.out.println("   MatchMap: " + mapEv.getMatchMap());
              System.out.println("   MatchMap alt. repr.: " + mapEv.getMatchMapList());
              System.out.println("   is Head?: " + mapEv.isHead());
              System.out.println("   is Overmatch?: " + mapEv.isOvermatch());
              System.out.println("   Sources: " + mapEv.getSources());
              System.out.println("   Positional Info: " + mapEv.getPositionalInfo());
            }
          }
    }
}
\end{verbatim}

Refer to the API javadoc for more information on the available methods
for each interface.


A complete example of this code is in
\texttt{src/javaapi/sources/gov/nih/nlm/nls/metamap/MetaMapApiTest.java}.




\subsection{Getting Raw MetaMap Machine Output}
\label{Getting Raw MetaMap Machine Output}

A copy of the raw MetaMap machine output can be obtained by using the
instance method \texttt{Result.getMachineOutput}:

\begin{verbatim}
List<Result> resultList = api.processCitationsFromString(terms);
Result result = resultList.get(0);
String machineOutput = result.getMachineOutput();
\end{verbatim}



\section{Advanced Configuration}
\label{Advanced Configuration}

\subsection{Running the metamap server on an alternate host}
\label{Running the metamap server on an alternate host}

To run the metamap server on a host other than localhost (127.0.0.1),
the default modify the environment variable ACCEPTED\_HOSTS the script
/bin/mmserver10 to contain all of the clients you wish to have access
to the MetaMap server.


For example, change the entry:

\begin{verbatim}
ACCEPTED_HOSTS="['127.0.0.1']"
export ACCEPTED_HOSTS
\end{verbatim}

to:

\begin{verbatim}
ACCEPTED_HOSTS="['127.0.0.1','192.168.111.27','192.168.111.61','192.168.111.76']"
export ACCEPTED_HOSTS
\end{verbatim}

all of the entries must be ip-addresses, hostnames will not work.


\subsection{Running the  metamap server on an alternate port}
\label{Running the  metamap server on an alternate port}

To run the metamap server on a port other than the default port (8066)
modify the environment variable ``MMSERVER\_PORT'' in the script bin/mmserver10.


E.G. to change the port to 8888:

\begin{verbatim}
MMSERVER_PORT=8888
export MMSERVER_PORT
\end{verbatim}

\subsection{Specifying alternate MetaMap Server hosts and ports in the API}
\label{Specifying alternate MetaMap Server hosts and ports in the API}

The MetaMap Java API now includes the methods \texttt{setHost} and \texttt{setPort}
to specify the host and port locations of the MetaMap server.  The
source to the class gov.nih.nlm.nls.metamap.MetaMapApiTest
(public\_mm/src/javaapi/sources/gov/nih/nlm/nls/metamap/
MetaMapApiTest.java) contains an example of the use of these methods.




\section{Possible enhancements}
\label{Possible enhancements}

Possible enchancments include providing a UIMA module that uses an
underlying MetaMapApi instance, providing a factory method for
instantiating MetaMapApi instances, and providing instantiation of
instances through JNDI.


\section{Differences compared to MMTx API}
\label{Differences compared to MMTx API}

The MetaMap Java API only provides access to components available
through machine output.  That includes The Final mappings, the
Candidates, Phrases, Utterances, Negated Concepts, and Acronyms and
Abbreviations.  Access to structures such as lexical elements and
tokenization that were available in the eariler MMTx API are not
currently available in the MetaMap API.


\section{For more information}
\label{For more information}
\begin{description}
\item[Quintus Prolog]  \url{http://www.sics.se/isl/quintus/html/quintus}/
\item[PrologBeans]  \url{http://www.sics.se/sicstus/docs/3.12.9/html/prologbeans}/
\item[MetaMap]  \url{http://metamap.nlm.nih.gov}/
\end{description}


\end{document}


